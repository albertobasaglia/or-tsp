\documentclass{article}
\usepackage{graphicx}
\usepackage{amsmath}
\usepackage{amsfonts}
\usepackage{algorithm}
\usepackage{algpseudocode}
\usepackage{caption}
\usepackage{subcaption}

\title{Performance Analysis of TSP Solving Algorithms}
\author{Basaglia Alberto, Stocco Andrea}
\date{\today}

\begin{document}

\maketitle

\begin{abstract}
	The Traveling Salesman Problem (TSP) is one of the most intensively studied problems in optimization.
	It asks the following question:\\\\
	\textit{Given a list of cities and the distances between each pair of cities, what is the shortest possible route 
	that visits each city exactly once and returns to the origin city?}\\\\
	TSP is computationally difficult (NP-Hard), but there exist many \textit{heuristics} and \textit{exact algorithms} to solve it.
	In this thesis we are going to show and compare the performances of these algorithms. 
\end{abstract}

\section{Introduction}
In this study we focus on evaluating the performance of two heuristic algorithms: the \textit{Greedy} algorithm with \textit{2-opt} 
optimization and the \textit{Tabu Search} algorithm.
The Greedy algorithm, starting from every possible node of the instance, tries to find the optimal solution always updating the path
with the closest node. To improve the performance of this algorithm, after that, we apply also the 2-opt algorithm, to remove 
the intersection between the edges of the path.
On the other hand, the Tabu Search algorithm, starting from a random node, tries to `move`... TODO

\section{Heuristics}
\subsection{Greedy}
\subsection{2-opt}

\section{Metaheuristics}
A metaheuristic is a versatile problem-solving approach characterized by its iterative nature and adaptability across various optimization problems. Unlike specific algorithms tailored to particular problems, metaheuristics serve as overarching strategies that guide subordinate heuristics to efficiently explore and exploit solution spaces. They intelligently combine different concepts to navigate through search spaces, aiming to find near-optimal solutions effectively.
In our experiments, we will adopt metaheuristic using \textit{2-opt} as the underlying
heuristic. We will prove that simple but very clever ideas (like Tabu Search and VNS) combined with the previously seen heuristics allow us to get solutions very close to the optimal ones.

\subsection{Tabu Search}
Tabu Search is a metaheuristic algorithm that efficiently explores the solution space by intelligently navigating through a neighborhood of solutions while maintaining a short-term memory to avoid revisiting previously visited or less promising solutions. The algorithm is particularly effective for combinatorial optimization problems like the Traveling Salesman Problem (TSP).

By design, this procedure will initially head directly towards a local minimum, thanks to the underlying heuristic. The only purpose of this metaheuristic is then to escape it and, hopefully, converge to a better minimum.

The search procedure will be alternating between 2 different behaviors by its nature. First of all
we will identify a phase where the method, starting from a ``bad'' solution, uses the underlying
heuristic to improve its cost. We will call this ``intensification phase''. Once a local minimum is reached, we will apply some bad moves to escape from this solution. We name this the ``diversification phase''. We hope that the alternation between improving the solution and moving
away from it allows us to explore different local minimum, allowing us to get an overall better
solution.

Algorithm~\ref{alg:tabu} shows a very abstract description of the procedure we use in practice. The first solution is found using a greedy method (the closest neighbor apporach) and then \textit{2-opt} is used for the intensification phase. It is important to notice that $delta$ is
how much the solution cost would improve (decrease by).

\begin{algorithm}[h]
\caption{Tabu Search}
\label{alg:tabu}
\begin{algorithmic}
  \Procedure{tabusearch}{$solution$}
    \State\Call{greedy}{solution}
    \State $tabulist \gets \emptyset$

    \While{$!stop$}
        \State $move \gets \Call{findbestswapnotabu}{solution, tabulist}$
        \State $delta \gets \Call{delta}{move}$
        \State $solution \gets \Call{apply}{solution, move}$
        \If{$delta \leq 0$}
          \State $tabulist \gets tabulist \cup \{move\}$
        \EndIf
        \State $\Call{removeold}{tabulist}$
    \EndWhile

  \EndProcedure

\end{algorithmic}
\end{algorithm}

\subsection{VNS (Variable Neighborhood Search)}
In \textit{Tabu Search} we have seen two different phases: 
\begin{itemize}
	\item the \textit{Intensification Phase} to move towards a better solution
	\item the \textit{Diversification Phase} in which we try to `escape` from a local minimum using only
	non-tabu moves
\end{itemize}
With this approach, we may spend a considerable amount of time without achieving any improvement in 
solution quality (Diversification Phase). One potential solution could involve restarting from a random 
point each time we encounter a local minimum (Multistart). However, this approach would significantly 
prolong the Intensification Phase. Moreover, when encountering a local minimum, it is probable 
that the solution obtained is not entirely incorrect but requires adjustment in some aspect.
VNS is a metaheuristic that employs minimal permutations of the solution to evade local minima, as 
opposed to restarting from a completely different starting point.
With this approach, we transition from one solution to a better one using the 2-opt method until reaching 
a local minimum. Subsequently, we perform a random number of permutations involving three nodes (3-opt)
to navigate away from the minimum. This approach also ensures that we never revert to a previously visited
solution in the last 2-opt application, as it cannot be achieved through a sequence of permutations of
three nodes.
This is a simplified version of the algorithm; state-of-the-art techniques also allow for larger permutations
of nodes (e.g., 4-opt, 5-opt, etc.) instead of multiple 3-opt operations.
Algorithm~\ref{alg:vns} shows a very abstract description of the procedure

\begin{algorithm}[h]
\caption{VNS}
\label{alg:vns}
\begin{algorithmic}
  \Procedure{vns}{$solution$}
    \State\Call{greedy}{solution}

    \While{$!stop$}
        \State $move \gets \Call{findbest2optswap}{solution}$
        \State $delta \gets \Call{delta}{move}$
        \If{$delta \leq 0$}
			\For{random number of times}
				\State $move \gets \Call{find3optswap}{solution}$
		        \State $solution \gets \Call{apply}{solution, move}$
			\EndFor
        \EndIf
        \State $solution \gets \Call{apply}{solution, move}$
    \EndWhile

  \EndProcedure

\end{algorithmic}
\end{algorithm}

\section{Exact Methods}
\section{Results}

\end{document}
